\section{Les plus courts chemins}
\subsection{Les plus courts chemins}
\index{graphe!graphe pondéré}
\begin{mydef}
  Une \emph{fonction de poids} sur un graphe ($V$, $E$, $\varphi$) est une fonction $E \to \mathbb{R}$. Un \emph{graphe pondéré} est un graphe muni d’une fonction de poids. Le \emph{poids} ou la \emph{longueur} d’un parcours est la somme des poids des arêtes qui le compose.
\end{mydef}

\begin{mytheo} [Plus court chemin et plus court parcours]
  Pour un graphe avec une fonction de poids $\geq 0$, si le plus court parcours entre $u$ et $v$ est de longueur $d$, alors le plus court chemin entre $u$ et $v$ est aussi de longueur $d$.
  \begin{proof}
     Cette démonstration n'a pas été vue au cours.
  \end{proof}
\end{mytheo}

\subsection{Algorithme de Dijkstra}
\index{algorithme!algorithme de Dijkstra}
\begin{myalgo}[Algorithme de Dijkstra]
\noindent
\begin{enumerate}
  \item Initialement, $i=0$, l'ensemble $S={u_{0}}$ (il contient notre noeud de départ) et l'ensemble \={S}$_{0}$ contient tous les autres noeuds. $l(u_{0})=0$ et $l(v)=\infty$ pour $v\ne u_{0}$ ($l(v)$ est une borne supérieure de $d(u_{0},v)$).
	\item Pour chaque $v \in$ \=S$_{i}$, on remplace $l(v)$ par min $({l(v),l(u_{i}+w{u_{i}v)}})$. On calcule le minimum pour $v \in$\=S$_{i}$ de $l(v)$ et on nomme $u_{i+1}$ le sommet pour qui ce minimum est atteind. Ensuite, on fixe $S_{i+1}=S_{i} \cup {u_{i+1}}$ et on retire ce sommet de \=S$_{i+1}$.
	\item si $i=v-1$, on arrête. Si $i < v-1$, on remplace i par $i+1$ et on refait l'étape 2.
	\end{enumerate}
\end{myalgo}

\begin{myexem}
  On cherche les distances à partir de $a$.\\
	\begin{center}
	  \begin{tikzpicture}
      \node[vertex] at (0,  2) (a) {$a$};
      \node[vertex] at (2,  1) (b) {$b$};
      \node[vertex] at (1, -2) (c) {$c$};
      \node[vertex] at (-1,-2) (d) {$d$};
      \node[vertex] at (-2, 1) (e) {$e$};
      \draw[->] (a) edge node[anchor = south] {\tiny $50$} (b);
  		\draw[->] (c) edge node[anchor = north] {\tiny $5$} (b);
  		\draw[->] (c) edge node[anchor = north] {\tiny $50$} (d);
  		\draw[->] (e) edge node[anchor = south] {\tiny $10$} (d);
  		\draw[->] (a) edge node[anchor = south] {\tiny $10$} (e);
  		\draw[->] (d) edge node[anchor = north] {\tiny $5$} (a);
  		\draw[->] (a) edge node[anchor = south] {\tiny $30$} (c);
  		\draw[->] (d) edge node[anchor = north] {\tiny $20$} (b);
  	\end{tikzpicture}
  \end{center}
  \begin{center}
    \begin{tabular}{|c|c|ccccc|}
      \hline
      $u'$ & $S$ & $l(a)$ & $l(b)$ & $l(c)$ & $l(d)$ & $l(e)$\\
      \hline
      $a$ & $\lbrace a \rbrace$ & $\boxed{0}$ & $\infty$ & $\infty$ & $\infty$ &$\infty$\\
      $a$ & $\lbrace a,e \rbrace$ && 50 & 30 & $\infty$ & $\boxed{10}$\\
      $e$ & $\lbrace a,e,d \rbrace$ && 50 & 30 & $\boxed{20}$ &\\
      $d$ & $\lbrace a,e,d,c \rbrace$ && 40 & $\boxed{30}$ & &\\
      $c$ & $\lbrace a,e,d,c,b \rbrace$ && $\boxed{35}$ & & &\\
      \hline
    \end{tabular}
  \end{center}
\end{myexem}

\begin{mytheo} [L'algorithme de Dijkstra fonctionne]
  Après chaque MISE A JOUR DE $\ell$ dans l’algorithme, les deux propriétés suivantes sont vérifiées :
  \begin{itemize}
    \item pour $v \in S$, $\ell(v) = d(u_0, v)$ et le chemin le plus court de $u_0$ à $v$ reste dans $S$;
    \item pour $v \notin S$, $\ell(v) \geq d(u_0, v)$, et $\ell(v)$ est la longueur du plus court chemin de $u_0$ vers $v$ dont tous les noeuds internes sont dans $S$.
  \end{itemize}
  \begin{proof}
     Cette démonstration n'a pas été vue au cours.
  \end{proof}
\end{mytheo}

\begin{mycorr} [L'algorithme de Dijkstra est correct]
  L’algorithme de Dijkstra est correct.
\end{mycorr}

\begin{mytheo} [L’algorithme de Dijkstra est quadratique]
  L’algorithme de Dijkstra sur un graphe se termine en un temps de l’ordre $n^2$ .
  \begin{proof}
    \noindent
    \begin{itemize}
		\item A chaque passage de boucle $\left\lbrace v \in S \right\rbrace $ décroit de 1. Ce qui implique $n$ passages de boucles.
		\item A chaque boucle, "$\forall v \in S$" fait $O(n-|S|)$ opérations; et "trouver $v_{min}$" fait $O(n-|S|)$ opérations également.
		\item Cela implique un temps total $O(n + (n-1) + (n-2) + ... + 1) = O\left( \frac{n(n-1)}{2} \right) = O(n^2)$ puisqu'on considère le temps à une constante près.
		\end{itemize}
  \end{proof}
\end{mytheo}

\index{graphe!graphe dirigé}
\begin{mydef}
  Un \emph{graphe dirigé} est un triplet ($V$, $E$, $\varphi$), où :
  \begin{itemize}
    \item $V$ est un ensemble dont les éléments sont appelés sommets ou noeuds;
    \item $E$ est un ensemble dont les éléments sont appelés arêtes;
    \item $\varphi$ est une fonction, dîte fonction d'incidence, qui associe à chaque arête un \emph{couple}
      \footnote{Dans la définition~\ref{def:graph} on a parlé de paire et ici de couple car une paire est
        un ensemble de 2 éléments alors qu'un couple est un $n$-uplet avec $n = 2$.
        Dans un ensemble, il n'y a pas de relation d'ordre alors que dans un $n$-uplet,
        chaque élément a un indice bien définit.
        Une arête est donc à sens unique dans un graphe dirigé.}
      de sommets.
  \end{itemize}
\end{mydef}

\begin{myexem}
Pour le graphe dirigé suivant :
  \begin{center}
	  \begin{tikzpicture}
      \node[vertex] at (0,  2) (a) {$a$};
      \node[vertex] at (2,  1) (b) {$b$};
      \node[vertex] at (1, -2) (c) {$c$};
      \node[vertex] at (-1,-2) (d) {$d$};
      \node[vertex] at (-2, 1) (e) {$e$};
      \draw[->] (a) edge node[anchor = south]   {\tiny $1$} (b);
  		\draw[->] (c) edge node[anchor = north] {\tiny $2$} (b);
  		\draw[->] (c) edge node[anchor = north] {\tiny $3$} (d);
  		\draw[->] (e) edge node[anchor = south] {\tiny $4$} (d);
  		\draw[->] (a) edge node[anchor = south] {\tiny $5$} (e);
  		\draw[->] (d) edge node[anchor = north] {\tiny $6$} (a);
  		\draw[->] (a) edge node[anchor = south] {\tiny $7$} (c);
  		\draw[->] (d) edge node[anchor = north] {\tiny $8$} (b);
  	\end{tikzpicture}
  \end{center}
	Nous avons :
	$V=\left\{a,b,c,d,e\right\}$\\
	$E=\left\{1,2,3,4,5,6,7,8\right\}$\\
	$\varphi(1)=\left\{a,b\right\}$\\
	$\varphi(2)=\left\{c,b\right\}$\\
	$\varphi(3)=\left\{c,d\right\}$\\
	$\varphi(4)=\left\{e,d\right\}$\\
	$\varphi(5)=\left\{a,e\right\}$\\
	$\varphi(6)=\left\{d,a\right\}$\\
	$\varphi(7)=\left\{a,c\right\}$\\
	$\varphi(8)=\left\{d,b\right\}$\\

\end{myexem}

\subsection{Anneaux et Semi-anneaux}
\index{anneau}
\index{anneau!semi-anneau}
\begin{mydef}
  Un \emph{semi-anneau} est un ensemble muni de deux opérations ($\oplus,\otimes$) et ses propriétés.
\end{mydef}

\begin{mydef}
  Un \emph{anneau} est un ensemble muni de trois opérations ($\oplus,\otimes,\ominus$) et ses propriétés.
\end{mydef}

\begin{mydef}
  Un \emph{corps} est un ensemble muni de quatre opérations ($\oplus,\otimes,\ominus,\oslash$) et ses propriétés.
\end{mydef}

Par exemple, une propriété peux être la distributivité :
$$a\otimes (b \oplus c) = (a \otimes b)\oplus(a \otimes c) $$
On peut par exemple définir des matrices avec un semi-anneau :
\begin{eqnarray}
  A \oplus B &=& (a_{ij} \oplus b_{ij})_{ij}\\
  A \otimes B &=& (\sum_k a_{ik} \otimes b_{kj})_{ij}
\end{eqnarray}
Le $\oplus$ définit la somme matricielle et le $\otimes$ le produit matriciel.\\
Pour calculer le plus court chemin, on peut définir l'itération :
\begin{eqnarray}
  M_0 &=& I\\
  M_1 &=& I \oplus (M_0 \otimes A)\\
  M_2 &=& I \oplus (M_1 \otimes A)\\
   &\vdots &\\
  M_{k+1} &=& I \oplus (M_k \otimes A)
\end{eqnarray}
On a fini de converger quand $M_k = M_{k+1}$. Dijkstra peut être vu comme une manière efficace d'implémenter cette itération.
